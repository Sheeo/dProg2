\documentclass{article}
\DeclareMathSymbol{,}{\mathpunct}{letters}{"3B}
\DeclareMathSymbol{.}{\mathord}{letters}{"3B}
\DeclareMathSymbol{\decimal}{\mathord}{letters}{"3A}
\usepackage[utf8]{inputenc}
\usepackage[danish]{babel}
\usepackage{amsmath}
\setlength{\parindent}{0in}
\setlength{\parskip}{0.1in}
\newcommand{\ArrayList}{\texttt{ArrayList}}
\newcommand{\List}{\texttt{List}}
\newcommand{\Set}{\texttt{Set}}
\newcommand{\HashSet}{\texttt{HashSet}}
\newcommand{\hashCode}{\texttt{hashCode}}
\newcommand{\equals}{\texttt{equals}}
\newcommand{\MultiSet}{\texttt{MultiSet}}
\author{Mathias Rav og Michael Søndergaard}
\title{dProg2 Uge 5}
\date{3.\ december 2010}
\begin{document}
\maketitle
\paragraph{Spørgsmål 1.}
Ved at bruge HashMap som "storage" i vores HashSet klasse, er det nemt for os
at implementere toString(), hashCode() og equals() ved bare at delegere kaldene
videre til HashMap.

\paragraph{Spørgsmål 2.}
Vi kan override addAll() metoden fra AbstractCollection, men den er ikke til meget
nytte, uden vi benytter den til at iterere igennem parameteren som vi får,
og tilføje hver enkelt element til vores MultiSet.




\end{document}
