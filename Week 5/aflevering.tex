\documentclass{article}
\DeclareMathSymbol{,}{\mathpunct}{letters}{"3B}
\DeclareMathSymbol{.}{\mathord}{letters}{"3B}
\DeclareMathSymbol{\decimal}{\mathord}{letters}{"3A}
\usepackage[utf8]{inputenc}
\usepackage[danish]{babel}
\usepackage{amsmath}
\setlength{\parindent}{0in}
\setlength{\parskip}{0.1in}
\newcommand{\Committee}{\texttt{Committee}}
\newcommand{\Employee}{\texttt{Employee}}
\newcommand{\Manager}{\texttt{Manager}}
\newcommand{\FindRoot}{\texttt{FindRoot}}
\newcommand{\ArrayList}{\texttt{ArrayList}}
\newcommand{\List}{\texttt{List}}
\newcommand{\Set}{\texttt{Set}}
\newcommand{\HashSet}{\texttt{HashSet}}
\newcommand{\hashCode}{\texttt{hashCode}}
\newcommand{\equals}{\texttt{equals}}
\author{Mathias Rav og Michael Søndergaard}
\title{dProg2 Uge 5}
\date{3.\ december 2010}
\begin{document}
\maketitle
\section{Aggregering af medlemmer i \Committee: \List{} eller \Set}
En \List{} repræsenterer en ordnet liste af elementer. Listen har en længde, og
dens elementer kan tilgås enten vilkårligt vha.\ \texttt{get()}-metoden eller i
rækkefølge vha.\ \texttt{Iterable}-interfacet. Det samme element kan optræde
flere gange i listen.

Et \Set{} repræsenterer en uordnet samling af elementer hvor hvert element kun
kan optræde \'en gang.

I forbindelse med en \Committee{} som har en række medlemmer virker det
umiddelbart oplagt at benytte et \Set{} i den konkrete implementation af et
\HashSet. Når vi skal tilføje et medlem til komitteen, skal vi ikke være
bekymrede for at det samme medlem bliver tilføjet to gange, for det håndterer
et \Set{} ved ikke at gøre noget. Havde vi valgt i stedet at bruge en \List,
såsom \ArrayList, så ville vi være nødt til at gennemløbe listen, hver gang vi
tilføjede et medlem, for at se om det i forvejen var tilføjet.

Derfor har vi valgt at indholde komitteens medlemmer i et \HashSet.

For at få den bedste ydeevne ud af et \HashSet{} kræver det, at klassen af
objekter, den indeholder, har en meningsfuld implementation af
\hashCode-metoden. Derudover kræver \HashSet{}, at hvert elements \hashCode{}
ikke ændres efter elementet er tilføjet til sættet. Ellers vil nogle metoder,
f.eks. \equals, returnere et forkert resultat.

Derfor har vi valgt at lade vores \Employee{} og \Manager{} være immutable.
\section{\FindRoot}
Metoden \texttt{FindRoot.bisection()} er numerisk ustabil i udregningen af
\texttt{average}. I floating point arithmetic er det bedre at udregne
gennemsnittet af \texttt{a} og \texttt{b} som \texttt{a+(b-a)/2} end som
\texttt{(a+b)/2}, da man nemt mister betydelige cifre i summen af $a$ og $b$.

Derudover er stopbetingelsen \texttt{f.getValue(average) == 0} skidt pga.\ den
direkte sammenligning mellem to floating point-tal. Denne stopbetingelse har vi
skrevet om, så TODO: beskriv

Derfor har vi omskrevet \texttt{bisection}. 
\end{document}
