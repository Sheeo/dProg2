\documentclass[12pt,a4paper]{article}
\usepackage[utf8]{inputenc}
\usepackage[danish]{babel}
\usepackage{amsmath,amssymb,amsthm}
\usepackage{listings}
\usepackage{color}
\lstloadlanguages{C,sh,JVMIS}
\setlength{\parindent}{0in}
\setlength{\parskip}{0.1in}

\begin{document}

\title{Programmering 2, E2010. Aflevering 3}
\author{Mathias Rav, 20103940 \\
		Michael Søndergaard, 20104223 \\
		DAT-3}
\date{Uge 3, November 19, 2010}
\maketitle


\section{Anonyme klasser og hjælpemetoder}
Istedet for anonyme klasser til brug som ActionListener's, har vi valgt at 
lave en subklasse af JButton, som vi har kaldt ColorButton. Vi kunne dog 
have brugt en hjælpemetode og anynom klasse til at opnå det samme.

Vi har valgt at gøre det på denne måde, fordi koden bliver pænere, og nemmere
at forstå. Hvis vi en dag skulle udvide programmet, ville det være nemt for os,
at genbruge koden, som vi allerede har skrevet, andre isteder i programmet. 

\section{Interfaces}
I vores program bruger vi to interfaces: \texttt{ActionListener} og \texttt{Icon} 

\section{Designmønstre}
Udover COMPOSITE-mønstret, benytter vi os af java-klasserne, som i sig selv 
implementerer en række designmønstre.

\end{document}