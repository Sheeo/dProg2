\documentclass[12pt,a4paper]{article}
\usepackage[utf8]{inputenc}
\usepackage[danish]{babel}
\usepackage{amsmath,amssymb,amsthm}
\usepackage{listings}
\usepackage{color}
\lstloadlanguages{C,sh,JVMIS}
\setlength{\parindent}{0in}
\setlength{\parskip}{0.1in}

\begin{document}

\title{Programmering 2, E2010. Aflevering 3}
\author{Mathias Rav, 20103940 \\
		Michael Søndergaard, 20104223 \\
		DAT-3}
\date{Uge 3, November 19, 2010}
\maketitle

\section{Anonyme klasser og hjælpemetoder}
I stedet for anonyme klasser til brug som ActionListeners, har vi valgt at 
lave en subklasse af JButton, som vi har kaldt ColorButton. Vi kunne dog 
have brugt en hjælpemetode og anynom klasse til at opnå det samme.

Vi har valgt at gøre det på denne måde, fordi koden bliver pænere, og nemmere
at forstå. Hvis vi en dag skulle udvide programmet, ville det være nemt for os,
at genbruge koden, som vi allerede har skrevet, andre isteder i programmet. 

\section{Interfaces}
I vores program bruger vi to interfaces. \texttt{BallIcon} og
\texttt{CompositeIcon} implementerer begge \texttt{Icon}, og vores
\texttt{ColorButton} implementerer \texttt{ActionListener}.

\section{Designmønstre}
Ud over \textit{Composite}, som vi har implementeret i \texttt{CompositeIcon},
lægger Javas indbyggede UI-biblioteker op til, at vi benytter \textit{Observer}
til at reagere på UI-hændelser, sådan som det er implementeret i
\texttt{ColorButton} med \texttt{ActionListener}-interfacet.

\textit{Strategy}-mønstret er brugt til at ændre farven på de enkelte
\texttt{ColorButtons} ved at udvide \texttt{BasicButtonUI}-klassen. Java
tilbyder mange måder at tegne knapper og andre UI-elementer på gennem sine
\texttt{ComponentUI}-klasser. For knapper findes der som standard både
\texttt{BasicButtonUI} og \texttt{MetalButtonUI}. Man kan nemt skifte udseendet
af sine knapper ved at kalde deres \texttt{setUI}-metode med den passende
\texttt{ButtonUI}-instans. I forhold til \textit{Strategy}-mønstret har vi
implementeret en ny strategi til konteksten at tegne knapper - eller rettere,
vi har udvidet \texttt{BasicButtonUI}-strategien til at passe til vores formål.

\texttt{IconCrafter} kunne have været opgraderet til en \textit{Facade}, som
stod for at håndtere interfacet til \texttt{JFrame} selv, for at gøre arbejdet
nemmere for driverklasserne \texttt{Driver} og \texttt{IconCrafterProgram}. Det
har dog ikke vist sig hverken nødvendigt eller bekvemt for et miniprojekt som
dette.

\end{document}
